\documentclass[11pt,a4paper]{article}

\usepackage[margin=1in, paperwidth=8.3in, paperheight=11.7in]{geometry}
\usepackage{amsfonts}
\usepackage{amsmath}
\usepackage{enumerate}
\usepackage{enumitem}
\usepackage{fancyhdr}
\usepackage{listings}
\usepackage{stmaryrd}
\usepackage[table]{xcolor}

\begin{document}

\pagestyle{fancy}
\setlength\parindent{0pt}
\allowdisplaybreaks

% Counters
\newcounter{question}
\newcounter{qpart}[question]
\newcounter{qqpart}[qpart]

% commands
\newcommand{\nats}{\mathbb{N}}
\newcommand{\reals}{\mathbb{r}}
\newcommand{\newquestion} {\stepcounter{question}}
\newcommand{\newqpart} {\stepcounter{qpart}}
\newcommand{\newqqpart} {\stepcounter{qqpart}}
\newcommand{\question}[2] {\newquestion \ifquestions \textbf{Question \arabic{question} - #1}\\ #2 \fi}
\newcommand{\qpart}[1] {\newqpart \ifquestions \textbf{Question \arabic{question}.\arabic{qpart}}\\ #1 \\ \fi}
\newcommand{\qqpart}[1] {\newqqpart \ifquestions \textbf{Question \arabic{question}.\arabic{qpart}.\arabic{qqpart}}\\ #1\\ \fi}
\newcommand{\solution}[1] {\ifsolutions\textbf{My Solution \arabic{question}}\\ #1\\ \fi}
\newcommand{\spart} {\ifsolutions\textbf{My Solution \arabic{question}.\arabic{qpart}} \fi}
\newcommand{\sspart} {\ifsolutions\textbf{My Solution \arabic{question}.\arabic{qpart}.\arabic{qqpart}} \fi}
\newcommand{\doubleplus} {+\kern-1.3ex+\kern0.8ex}
\renewcommand{\headrulewidth}{0pt}

% enviroments
\lstnewenvironment{code}
  {\lstset{mathescape=true,xleftmargin=.1\textwidth}}
  {}

% if
\newif\ifquestions
\questionstrue
%\questionsfalse
\newif\ifsolutions
\solutionstrue
%\solutionsfalse

% Cover page title
\title{Language Engineering - Problem Sheet 3}
\author{Dom Hutchinson}
\date{\today}
\maketitle

% Header
\fancyhead[L]{Dom Hutchinson}
\fancyhead[C]{Language Engineering - Problem Sheet 3}
\fancyhead[R]{\today}

\question{Simple Expressions}{
Consider the following simple expression language consisting of just values and additions.
}
\begin{code}
data Expr = Val Int
          | Add Expr Expr
\end{code}

\qpart{
Write a recursive funciton $eval::Expr\to Int$ which evaluates these arithmetic expressions.
}

\spart
\begin{code}
eval :: Expr $\to$ Int
eval (Val n) = n
eval (Add x y) = (eval x) + (eval y)
\end{code}

\qpart{}
\qqpart{
Define the datatype $Fix\ f$ with a single constructor called $In::f\ (Fix\ f)\to Fix\ f$. Also define $inop::Fix\ f\to f\ (Fix\ f)$ which unwraps one layer of structure.
}

\sspart
\begin{code}
data Fix f = In (f (Fix f))

inop :: Fix f -> f (Fix f)
inop (In x) = x
\end{code}

\qqpart{
Define a new datatype $ExprF\ k$ which mirrors the constructors of $Expr$ except that the parameter $k$ replaces recursive occurences of $Expr$.
}

\sspart
\begin{code}
data ExprF k = ValF Int
             | AddF k k
\end{code}

\qqpart{
Let $Expr'$ be a type alias for $Fix\ ExprF$. Write down three values of type $Expr'$.
}

\sspart
\begin{code}
ex1 = In(ValF 1)
ex2 = In(ValF 4)
ex3 = In(AddF (In (ValF 1)) (In (ValF 2)))
\end{code}

\qqpart{
Define the recursive function $fromExpr::Expr\to Expr'$ that converts values from $Expr$ to $Expr'$.
}

\sspart
\begin{code}
fromExpr :: Expr -> (Fix ExprF)
fromExpr (Val n) = In (ValF n)
fromExpr (Add x y) = In(AddF (fromExpr x) (fromExpr y))
\end{code}

\qpart{}
\qqpart{
Define a function $cata::Functor\ f\Rightarrow(f\ a\to a)\to Fix\ f\to a$ which reduces the $Fix\ f$ structure into a single value of type $a$ using the provided algebra of type $f\ a\to a$.
}

\sspart
\begin{code}
cata :: Functor f => (f a -> a) -> Fix f -> a
cata alg (In x) = alg (fmap (cata alg) x)
\end{code}

\qqpart{
Give a $Functor$ instance for your $ExprF$ type.
}

\sspart{
\begin{code}
instance Functor ExprF
  where
    fmap f (ValF n) = ValF n
    fmap f (AddF x y) = AddF (f x) (f y)
\end{code}

\qqpart{
Write a function $eval'::Expr'\to Int$ using the function $cata$ and an appropriate algebra.
}

\sspart
\begin{code}
eval' :: Fix ExprF-> Int
eval' = cata alg
  where
    alg :: ExprF Int -> Int
    alg (ValF n) = n
    alg (AddF x y) = x+y
\end{code}

\qqpart{
Discuss the advantages of using $cata$ to define the evaluation function for the $Expr'$ type versus the original $eval$ for $Expr$.
}

\sspart\\
TODO\\

\qpart{}
\qqpart{
Define the function $toExpr::Expr'\to Expr$ that converts values from $Expr'$ to $Expr$. This must not be a recursive function.
}

\sspart
\begin{code}
toExpr :: Fix ExprF -> Expr
toExpr = cata alg
  where
    alg :: ExprF (Expr) -> Expr
    alg (ValF n) = Val n
    alg (AddF x y) = Add x y
\end{code}

\qqpart{
An isomorphism between two types $A$ and $B$, written $A\cong B$, exists when there are functions $f:A\to B$ and $g:B\to A$ such that $f\circ g=id$ and $g\circ f=id$. Prove that $Expr\cong Expr'$.
}

\sspart
\begin{code}
TODO
\end{code}

\question{Composing Expressions}{
It is possible to take the coproduct of functors $f$ and $g$ as given by the following type. This is a type operator that we have declared ot be right associative with precedence $5$.
}
\begin{code}
data (f :+: g) a = L (f a)
                 | R (g a)
infixr 5 :+:
\end{code}

\qpart{
Give the $Functor$ instance for $f:+:g$ under teh assumption that $f$ and $g$ are functors.
}

\spart
\begin{code}
instance (Functor f, Functor g) => Functor (f:+:g)
  where
    fmap f (L x) = L (fmap f x)
    fmap f (R y) = R (fmap f y)
\end{code}

\qpart{
The datatpe $ExprF$ you defined in the last section can be decomposed into two parts: $ValF\ k$ and $AddF\ k$ which represent the abstract syntax for values and addition respectively.
}

\qqpart{
Give the definitions of both $ValF$ and $AddF$.
}

\sspart
\begin{code}
  data ValF k = ValF k
  data AddF k = AddF k k
\end{code}

\qqpart{
Give $Functor$ instances for both $ValF$ and $AddF$.
}

\sspart
\begin{code}
instance Functor ValF
  where
    fmap f (ValF n) = ValF (f n)

instance Functor AddF
  where
    fmap f (AddF x y) = AddF (f x) (f y)
\end{code}

\qpart{
Write a new datatype $SubF\ k$ which represents the abstract syntax for subtraction. Furthermore, write a $Functor$ instance for $SubF$.
}

\spart
\begin{code}
data SubF k = SubF k k

instance Functor SubF
  where
    fmap f (SubF x y) = SubF (f x) (f y)
\end{code}

\qpart{
Write a function $evalAddSub::Fix(ValF:+:AddF:+:SubF)\to Int$ using $cata$ which can evaluate arithmetric expressions containing values, additions and subtractions.\\
\textit{Hint} - Look at the associativity of the $:+:$ operator.
}

\spart
\begin{code}
($\nabla$) :: (f a -> a) -> (g a -> a) -> (f :+: g) a -> a
(falg $\nabla$ galg) (L x) = falg x
(falg $\nabla$ galg) (R x) = galg x


val :: ValF Int -> Int
val (ValF x) = x

add :: AddF Int -> Int
add (AddF x y) = x + y

sub :: SubF Int -> Int
sub (SubF x y) = x - y

evalAddSub :: Fix (ValF :+: AddF :+: SubF) -> Int
evalAddSub = cata (val $\nabla$ (add $\nabla$ sub))
\end{code}

\qpart{
Explain why this method of composing datatypes will become more cumbersome as more datatypes are added into the composition.
}

\spart\\
TODO\\

\question{Classy Algebras}{
Consider the new typeclass $Alg\ f\ a$ which contains the operation of an algebra, $alg::f\ a\to a$, where $f$ is a functor and $a$ is the carrier.\\
This class requires multiple parameters.
}
\begin{code}
class Functor f $\Rightarrow$ Alg f align
  where
    alg :: f a $\to$ a
\end{code}

\qpart{
Give an instance for $Alg$ for $ValF$ with carrier $Int$.
}

\spart
\begin{code}
instance Alg ValF Int
  where
    -- alg :: ValF Int -> Int
    alg (ValF x) = x
\end{code}

\qpart{
Give an instance for $Alg$ for $AddF$ with carrier $Int$.
}

\spart
\begin{code}
instance Alg AddF Int
  where
    alg (AddF x y) = x + y
\end{code}

\qpart{
Give an instance for $Alg$ for $SubF$ with carrier $Int$.
}

\spart
\begin{code}
instance Alg SubF Int
  where
    alg (SubF x y) = x - y
\end{code}

\qpart{
Give the instance for $Alg\ (f:+:g)\ a$, assuming that instances of $Alg\ f\ a$ and $Alg\ g\ a$ exist.
\textit{Hint} - Use class constraints on your instance.
}

\spart{
\begin{code}
instance (Alg f a, Alg g a) => Alg (f :+: g) a
  where
    -- alg :: (f :+: g) a -> a
    alg (L x) = alg x
    alg (R y) = alg y
\end{code}

\qpart{
Now give a new definition for $evalAddSub::Fix\ (ValF:+:AddF:+:SubF)\to Int$ using $cata$ and an appropriate algebra.\\
\textit{Hint} - Your answer should use the algebra from the $Alg$ class.
}

\spart
\begin{code}
evalAddSub' :: Fix (ValF :+: AddF :+: SubF) -> Int
evalAddSub' = cata alg
\end{code}

\qpart{
Define a function $cati::Alg\ f\ a\Rightarrow Fix\ f\to a$ which performs teh same role as $cata$, but using the $alg$ given by $Alg\ g\ a$ instance.
}

\spart
\begin{code}
cati :: Alg f a => Fix f -> a
cati (In x) = alg (fmap cati x)
\end{code}

\qpart{}
\qqpart{
Extend the expression language you have built so far by including a new datatype $MulF$.
}

\sspart
\begin{code}
data MulF k = MulF k k

instance Functor MulF
  where
    fmap f (MulF x y) = MulF (f x) (f y)

mul :: MulF Int -> Int
mul (MulF x y) = x * y
\end{code}

\qqpart{
Define an appropriate algebra with carrier $Int$ that performs the multiplication.
}

\sspart
\begin{code}
instance Alg MulF Int
  where
    alg (MulF x y) = x * y
\end{code}

\qqpart{
Redefine $Expr$ to be a synonym for the expression language that includes multiplication.
}

\sspart
\begin{code}
type ExprF = Fix (ValF :+: AddF :+: MulF :+: SubF)
\end{code}

\qqpart{
Use $cati$ to define a function $eval :: Expr\to Int$ that evaluates expressions.
}

\sspart
\begin{code}
eval :: ExprF -> Int
eval = cati
\end{code}

\qpart{
Without changing any previous code, write a function $depth::Expr\to Int$ which returns the depth of the deepest node in an expression.\\
\textit{Hint} - Use a newtype for Int to provide a specialisation carrier.
}

\spart
\begin{code}
TODO
\end{code}

\end{document}
