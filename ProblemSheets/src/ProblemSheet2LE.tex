\documentclass[11pt,a4paper]{article}

\usepackage[margin=1in, paperwidth=8.3in, paperheight=11.7in]{geometry}
\usepackage{amsfonts}
\usepackage{amsmath}
\usepackage{enumerate}
\usepackage{enumitem}
\usepackage{fancyhdr}
\usepackage{listings}
\usepackage{stmaryrd}
\usepackage[table]{xcolor}

\begin{document}

\pagestyle{fancy}
\setlength\parindent{0pt}
\allowdisplaybreaks

% Counters
\newcounter{question}
\newcounter{qpart}[question]
\newcounter{spart}[question]

% commands
\newcommand{\nats}{\mathbb{N}}
\newcommand{\newquestion} {\stepcounter{question}}
\newcommand{\newqpart} {\stepcounter{qpart}}
\newcommand{\newspart} {\stepcounter{spart}}
\newcommand{\question}[2] {\newquestion \ifquestions \textbf{Question 1.\arabic{question} - #1}\\ #2\\ \fi}
\newcommand{\qpart}[1] {\newqpart \ifquestions \textbf{Question 1.\arabic{question}.\arabic{qpart}}\\ #1\\ \fi}
\newcommand{\solution}[1] {\ifsolutions\textbf{My Solution \arabic{question}}\\ #1\\ \fi}
\newcommand{\spart}[1] {\newspart\ifsolutions\textbf{My Solution \arabic{question}.\arabic{spart}}\\ #1\\ \fi}
\newcommand{\codesol} {\newspart\ifsolutions\textbf{My Solution \arabic{question}.\arabic{spart}}\fi}

\newcommand{\doubleplus} {+\kern-1.3ex+\kern0.8ex}
\renewcommand{\headrulewidth}{0pt}

% enviroments
\lstnewenvironment{code}
  {\lstset{mathescape=true}}
  {}

% if
\newif\ifquestions
\questionstrue
%\questionsfalse
\newif\ifsolutions
\solutionstrue
%\solutionsfalse

% Cover page title
\title{Language Engineering - Problem Sheet 2}
\author{Dom Hutchinson}
\date{\today}
\maketitle

% Header
\fancyhead[L]{Dom Hutchinson}
\fancyhead[C]{Language Engineering - Problem Sheet 2}
\fancyhead[R]{\today}

\question{Introduction}{
Consider a domain-specific language for directed graphs. A \textit{vertex} is any natural number. An \textit{edge} consists
of a \textit{source} and a \textit{target} vertex. A \textit{graph} is a set of vertices $V$ and a bag of edges $E$, where all the vertices
in the edges $E$ are contained in $V$ . A bag (sometimes called a multiset) is a structure like set except that
duplicate elements are allowed.\\
Graphs can be composed out of four operations, empty, vertex , connect, and overlay, described as follows.
\begin{enumerate}[label=-]
  \item $empty :: Graph$ where empty is a graph that contains no vertices and no edges.
  \item $vertex :: Int → Graph$, where vertex v is a graph containing the vertex v, and no edges.
  \item $overlay :: Graph → Graph → Graph$, where overlay x y is a graph whose vertices are the union of the
vertices in x and the vertices in y, and whose edges are the edges in x followed by the edges in y.
  \item $connect :: Graph → Graph → Graph$, where the vertices in connect x y are all those in overlay x y,
and the edges are the edges in overlay x y followed by an edge from each vertex in x to each vertex in
y.
\end{enumerate}
Together, these combinators make it possible to express all kinds of graph structures.
}

\question{Deep Embedding} {
A deep embedding uses a datatype to represent a domain-specific language, where each constructor of the
datatype corresponds to an operation. These constructors are called core constructors of the language.
}

\qpart{
Define a datatype $Graph$ that encodes a deep embedding of the graph language. Introduce a core
constructor for each of the four operations.
}

\codesol
\ifsolutions
\begin{code}
data Graph = Empty
           | Vertex Int
           | Overlay Graph Graph
           | Connect Graph Graph
\end{code}
\fi

\qpart{
Define a function $vertices :: Graph \to [Int]$, where vertices g is a list of the vertices in the graph g.
}

\codesol
\ifsolutions
\begin{code}
vertices :: Graph -> [Int]
vertices (Empty)         = []
vertices (Vertex n)      = [n]
vertices (Overlay g1 g2) = nub ((vertices g1) ++ (vertices g2))
vertices (Connect g1 g2) = nub ((vertices g1) ++ (vertices g2))
\end{code}
\fi

\qpart{
Define a function $edges :: Graph \to Int$, where edges $g$ is the number of edges in the graph $g$. Hint: You
may need to use your vertices function.
}

\codesol
\ifsolutions
\begin{code}
edges :: Graph -> Int
edges (Empty)         = 0
edges (Vertex n)      = 0
edges (Overlay g1 g2) = (edges g1) + (edges g2)
edges (Connect g1 g2) = (edges g1) + (edges g2)
                      + (length(vertices g1) * length(vertices g2))
\end{code}
\fi

\qpart{
Define a function $roots :: Graph \to [Int]$, where roots $g$ is a list of the roots of the graph $g$. A root of a
graph $g$ is a vertex of $g$ which does not appear as the target of any edge in $g$.
}

\codesol
\ifsolutions
\begin{code}
roots :: Graph -> [Int]
roots (Empty)         = []
roots (Vertex n)      = [n]
roots (Overlay g1 g2) = nub (removeAll (intersect (vertices g1) (vertices g2))
                        (nub ((roots g1) ++ (roots g2))))
                        ++(intersect (roots g1) (roots g2))
roots (Connect g1 g2) = removeAll (intersect (vertices g1) (vertices g2))
                        (roots g1)

- - removes given element from list
remove :: Int -> [Int] -> [Int]
remove i [] = []
remove i (x:xs)
  | i == x    = remove i xs
  | otherwise = x : (remove i xs)

- - remove list of elements from list
removeAll :: [Int] -> [Int] -> [Int]
removeAll [] ys = ys
removeAll (x:xs) ys = removeAll xs (remove x ys)
\end{code}
\fi

\question{Shallow Embedding}{}

\qpart{
Provide a shallow embedding that produces the same semantics as \textit{vertices} by redefining an appropriate type for \textit{Graph}, and defining the behaviour of \textit{empty}, \textit{vertex} , \textit{overlay}, and \textit{connect}.
}

\codesol
\ifsolutions
\begin{code}
type Vertices = [Int]
type Graph' = Vertices

empty :: Graph
empty = []

vertex :: Int -> Graph
vertex n = [n]

overlay :: Graph -> Graph -> Graph
overlay g1 g2 = nub (g1++g2)

connect :: Graph -> Graph -> Graph
connect g1 g2 = nub (g1++g2)
\end{code}
\fi

\qpart{
Redefine \textit{Graph} again, as well as all the operations, to produce the same semantics as \textit{edges}. Hint: Your semantic domain should be a pair of values, one containing the information for \textit{vertices}, and the other for \textit{edges}.
}

\codesol
\ifsolutions
\begin{code}
type Edges = Int
type Graph = (Edges, Vertices)

empty :: Graph
empty = (0,[])

vertex :: Int -> Graph
vertex n = (0, [n])

overlay :: Graph -> Graph -> Graph
overlay (e1, v1) (e2,v2) = (e1+e2, nub (v1++v2))

connect :: Graph -> Graph -> Graph
connect (e1, v1) (e2, v2) = (e1+e2+(length v1 * length v2), nub(v1++v2))
\end{code}
\fi

\qpart{
Discuss why the shallow definition of \textit{edges} more efficient than the deep definition of \textit{edges}.
}

\spart{
Using the shallow definition we don't need to recalculate the vertices of a graph each time we want to referrence it.
}

\question{Classy Embedding}{
Instead of redefining \textit{Graph} for each semantics, it is better to provide a type class that captures all the
shallow operations, and use instances to provide each of the different semantics.
}

\qpart{
Define an appropriate type class called \textit{Graphy}, and show how the semantics of \textit{vertices} and \textit{edges} can be given using new types called \textit{Vertices} and \textit{Edges} respectively.
}

\codesol
\ifsolutions
\begin{code}
  class Graphy a where
    empty  :: a
    vertex  :: Int -> a
    overlay  :: a -> a -> a
    connect  :: a -> a -> a
\end{code}
\fi

\qpart{
Provide an instance of your type class that allows a semantics of the shallow embedding to be the deep
embedding. Hint: this is the instance \textit{Graphy Graph}.
}

\codesol
\ifsolutions
\begin{code}
instance Graphy Vertices where
  empty = Vertices []
  vertex x = Vertices [x]
  overlay (Vertices g1) (Vertices g2) = Vertices (nub (g1++g2))
  connect (Vertices g1) (Vertices g2) = Vertices (nub (g1++g2))

instance Graphy Edges where
  empty = Edges (0,[])
  vertex x = Edges (0,[x])
  overlay (Edges (e1,v1)) (Edges (e2,v2)) = Edges (e1+e2, nub(v1++v2))
  connect (Edges (e1,v1)) (Edges (e2,v2)) = Edges (e1+e2+(length v1 * length v2), nub(v1++v2))

instance Graphy Graph where
  empty = Empty
  vertex x = Vertex x
  overlay g1 g2 = Overlay g1 g2
  connect g1 g2 = Connect g1 g2
\end{code}
\fi

\question{Smart Constructors}{
A \textit{smart constructor} is a function that provides new operations which are defined in terms of existing ones.
}

\qpart{
Extend the language with a new smart constructor called $ring :: Graphy\ graph \to [Int] \to graph$ which
produces a graph where all the elements in the list are converted into vertices and connected into a
complete cycle connecting each element to the next, and the last to the first.
}

\codesol
\ifsolutions
\begin{code}
TODO
\end{code}
\fi

\qpart{
Instead of defining this function in terms of existing operations, consider the implications of adding a
new core constructor \textit{Ring} to the \textit{Graph} data type.\\
a) Outline the code that needs to be changed when adding a core constructor rather than a smart
constructor.\\
b) Explain the benefits of using smart constructors over core constructors when new semantic functions
are added to a given language.\\
c) Explain why using core constructors can potentially lead to more efficient implementations.
}

\spart{
TODO
}

\question{Comparing Approaches}{}

\qpart{
Discuss the changes required in order to add a new semantics of a deep embedding.
}

\spart{
To add a semantics to a deep embedding only requires the definition of a new function.
}

\qpart{
Discuss the changes required in order to add a new operation to a deep embedding.
}

\spart{
To add a new operation requires the data types to be updated and all functions which use these datatypes to be updated as well.
}

\qpart{
Discuss the changes required in order to add a new semantics of a shallow embedding.
}

\spart{
To add a new semantic to shallow embedding requires all functions to be redefined, or updated to now calculate a tuple output which includes the new semantics.
}

\qpart{
Discuss the changes required in order to add a new operation to a shallow embedding.
}

\spart{
Add a new operation requires only the definition of a new function.
}

\qpart{
Discuss the advantages gained by using type classes for a shallow embedding.
}

\spart{
Type classes allows you to define a different instance of each semantics, rather than only have one semantics or using tuples.
}

\qpart{
Discuss the advantages gained by using smart constructors for a deep embedding.
}

\spart{
A smart constructor allows you to add new operations without having to redefined existing ones, as it is defined in terms of existing operations.
}

\question{Reinterpretting Graphs}{
The graph language can be used to construct a graph that is interpreted traditionally either as an adjacency
list or as an adjacency matrix.
}

\qpart{
Write a function $lists :: Graph \to Map\ Int\ [Int]$ that realises the deep \textit{Graph} DSL as a traditional
adjacency list implementation.
}

\codesol
\ifsolutions
\begin{code}
lists :: Graph -> Map Int [Int]
lists Empty = Map.empty
lists (Vertex n) = Map.singleton n []
lists (Overlay g1 g2) = Map.union (lists g1) (lists g2)
lists (Connect g1 g2) = Map.map (nub) (Map.union (Map.map (++ (Map.keys (lists g2))) (lists g1)) (lists g2))
\end{code}
\fi

\qpart{
Write a new typeclass instance for \textit{Graphy} that realises the shallow \textit{Graph} DSL as a traditional adjacency
list implementation.
}

\codesol
\ifsolutions
\begin{code}
TODO
\end{code}
\fi

\qpart{
Discuss why it is harder to write the interpretation using the shallow embedding rather than the deep
embedding.
}

\spart{
TODO
}

\qpart{
Write a function $mat :: Graph \to MatGraph$ that realises the \textit{Graph} DSL as a traditional adjacency
matrix implementation (you can define the datatype \textit{MatGraph} however you want).
}

\codesol
\ifsolutions
\begin{code}
type MatGraph = ([Int], [[Int]])

-- Treates dupes as seperate vertices
mat :: Graph -> MatGraph
mat Empty = ([], [])
mat (Vertex n) = ([n], [[0]])
mat (Overlay g1 g2) = (v1++v2, (overlay' (length e1) e1 (length e2) e2))
  where
    v1 = fst (mat g1)
    v2 = fst (mat g2)
    e1 = snd (mat g1)
    e2 = snd (mat g2)
mat (Connect g1 g2) = (v1++v2, connect' (length e1) e1 (length e2) e2)
  where
    v1 = fst (mat g1)
    v2 = fst (mat g2)
    e1 = snd (mat g1)
    e2 = snd (mat g2)

-- Pads edges with zeroes
--          len(e1)   e1     len(e2)    e2          out
overlay' :: Int -> [[Int]] -> Int -> [[Int]] -> [[Int]]
overlay' _ [] _ [] = []
overlay' n [] m (y:ys) = ((zeroes n)++y) : (overlay' n [] m ys)
overlay' n (x:xs) m ys = (x++(zeroes m)) : (overlay' n xs m ys)

-- Creates list of n zeroes
zeroes :: Int -> [Int]
zeroes 0 = []
zeroes n = 0:(zeroes (n-1))

-- pads edges with 1s
connect' :: Int -> [[Int]] -> Int -> [[Int]] -> [[Int]]
connect' _ [] _ [] = []
connect' n [] m (y:ys) = ((zeroes n)++y) : (overlay' n [] m ys)
connect' n (x:xs) m ys = (x++(ones m)) : (overlay' n xs m ys)

-- creates list of n ones
ones :: Int -> [Int]
ones 0 = []
ones n = 1:(ones (n-1))
\end{code}
\fi

\end{document}
